\documentclass[a4paper]{article}

\usepackage[utf8x]{inputenc}
\usepackage[T1]{fontenc}
\usepackage[francais]{babel}
\usepackage{graphicx}
\usepackage{float}

\usepackage[titletoc,toc,title]{appendix} 
\usepackage{fullpage}

\title{\bsc{INSA} de Rennes \\ Quatrième année - Informatique \\ \bigskip \hrule \bigskip Projet d'Analyse, Conception et POO \\ \bigskip Rapport de conception \bigskip \hrule}
\author{Mickaël \bsc{Olivier}, Benoit \bsc{Travers}}

\begin{document}

\maketitle
\thispagestyle{empty}
\newpage

\tableofcontents
\thispagestyle{empty}
\newpage

\section*{Introduction}

\newpage



\section{Cas d'utilisation}

\begin{figure}[H]
\centering
\includegraphics[width=1\textwidth]{usecase_global.png}
\caption{\label{fig:global_usecase}Diagramme de cas d'utilisation global}
\end{figure}

\subsection{Création de partie}

\begin{figure}[H]
\centering
\includegraphics[width=1\textwidth]{usecase_creation.png}
\caption{\label{fig:creation_usecase}Diagramme de cas d'utilisation : Création de la partie}
\end{figure}

\subsection{Tour de jeu}

\begin{figure}[H]
\centering
\includegraphics[width=1\textwidth]{usecase_tour.png}
\caption{\label{fig:tour_usecase}Diagramme de cas d'utilisation : Tour de Jeu}
\end{figure}



\newpage

\section{Diagrammes de classe}
	
\subsection{Modélisation globale du jeu}

Annexe \ref{app:diagramme_classe}

\subsection{Patrons utilisés}

\begin{figure}[H]
\centering
\includegraphics[width=1\textwidth]{fabrique.png}
\caption{\label{fig:fabrique}Fabrique}
\end{figure}

\begin{figure}[H]
\centering
\includegraphics[width=1\textwidth]{monteur.png}
\caption{\label{fig:monteur}Monteur}
\end{figure}

\begin{figure}[H]
\centering
\includegraphics[width=1\textwidth]{poids_mouche.png}
\caption{\label{fig:poids_mouche}Poids-Mouche}
\end{figure}

\begin{figure}[H]
\centering
\includegraphics[width=0.7\textwidth]{strategie.png}
\caption{\label{fig:strategie}Stratégie}
\end{figure}



\newpage

\section{Diagramme d'états-transitions}

\begin{figure}[H]
\centering
\includegraphics[width=1\textwidth]{diagramme_etats-transitions.png}
\caption{\label{fig:etats-transitions}Diagramme d'états-transitions}
\end{figure}


\newpage

\section{Diagrammes d'intéraction}

\subsection{Création de partie}

Annexe \ref{app:creation_partie}

\begin{figure}[H]
\centering
\includegraphics[width=1\textwidth]{addPeupleA.png}
\caption{\label{fig:addPeupleA}Diagramme de séquence : Ajouter le peuple du joueur A}
\end{figure}

\begin{figure}[H]
\centering
\includegraphics[width=0.6\textwidth]{addCarte.png}
\caption{\label{fig:addCarte}Diagramme de séquence : Ajouter la carte}
\end{figure}

\subsection{Déplacer ou attaquer une unité}

Annexe \ref{app:tour}



\newpage

\section*{Conclusion}

\newpage

\begin{appendices}

\section{Diagramme de classes} \label{app:diagramme_classe}

\begin{figure}[H]
\centering
\includegraphics[width=0.7\textwidth]{diagramme_classes.png}
\end{figure}

\newpage
\section{Diagramme de séquence : Création de partie} \label{app:creation_partie}

\begin{figure}[H]
\centering
\includegraphics[width=\textwidth]{creation.png}
\end{figure}

\newpage
\section{Diagramme de séquence : Déplacer ou attaquer une unité} \label{app:tour}

\begin{figure}[H]
\centering
\includegraphics[width=0.65\textwidth]{tourDeJeu.png}
\end{figure}

\end{appendices}

\newpage
\listoffigures

\end{document}
